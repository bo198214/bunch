%Author: Henryk Trappmann, created 20080622

\begin{proposition}
  Let $b>1$ then for each integer $k\ge 2$ there is exactly one solution
  $z$ of $z=b^z$ in the horizontal strip $2\pi
  (k-1)/\ln(b)\le\Im(z)<2\pi k/\ln(b)$. We call this solution
  $b[k]$. More specifically it is
  situated in $2(k-1)\pi/\ln(b) < \Im(z) < (2\pi k -\pi)/\ln(b)$. For $k=1$ we distinguish 3 cases :
  \begin{enumerate}
  \item If $b>e^{1/e}$ then the above is also valid for $k=1$.
  \item
  If $b=e^{1/e}$ then there is exactly one solution for $k=1$,
  this solution is $e=:b[1]$;
  \item
  If $1<b<e^{1/e}$ then there are exactly two solutions
  $b^-<e<b^+=:b[1]$.
  \end{enumerate}
  For each solution $b[j]$, $j\ge 1$, the conjugate is also a
  solution of the equation which we denote by $b[-j]$. There are no
  other solutions than the before mentioned.
\end{proposition}
\begin{proof}
  Let $z=re^{i\alpha}=r(\cos(\alpha)+i\sin(\alpha))$ and let
  $c=\ln(b)$ then the fixed point equation is equivalent to the
  equation system:
  \begin{align}
    r&=e^{cr\cos(\alpha)}\label{eq:radius}\\
    \alpha&=cr\sin(\alpha)\label{eq:angle}
  \end{align}
  Let us now substitute $s=cr$ and assume $\alpha\neq 2\pi m$,
  for any integer $m\ge 0$:
  \begin{align*}
    \ln(s)-\ln(c)=\ln(r) &= s\cos(\alpha)\\
    s&=\frac{\alpha}{\sin(\alpha)}
  \end{align*}
  and substituting the second into the first
  \begin{align*}
    \ln\frac{\alpha}{\sin(\alpha)}-\ln(c) &= \alpha
    \frac{\cos(\alpha)}{\sin(\alpha)}\\
    f(\alpha):=\ln\frac{\alpha}{\sin(\alpha)}-\alpha\cot(\alpha)&=\ln(c)
  \end{align*}
  We show that $f$ is strictly increasing on $2k\pi< \alpha <
  (2k+1)\pi$ for $k\ge 0$ and strictly decreasing for $k<0$ by
  contemplating the sign of its derivative:
  \begin{align*}
    f'(x) &= \frac{\sin(x)}{x}\left( -\frac{x \cos(x)}{\sin(x)^2}
      + \frac{1}{\sin(x)} \right) 
    -\frac{\cos(x)}{\sin(x)} - x\left(-\sin(x)\frac{1}{\sin(x)}
      +\cos(x)\frac{-1}{\sin(x)^2}\cos(x) \right)\\
    &= - 2\cot(x) +\frac{1}{x}+x + x\cot(x)^2 = 
     x+\frac{-2 x \cot(x) + 1 + x^2 \cot(x)^2}{x}\\
    &= x+\frac{(1- x \cot(x))^2 }{x}
  \end{align*}
  The last line is positive for $x>0$ and negative for $x<0$, so there
  can be at most one solution of $f(x)=\ln(c)$ on $x\in(2 k\pi, 2 k\pi + \pi)$
  and there is a solution for $k\neq 0,-1$ because in this case
  $f((2k\pi,(2k+1)\pi))=(-\infty,+\infty)$.
  The imaginary part of $z$ for this solution is $r \sin(\alpha)$
  which is equal to $\alpha/\ln(b)$ by \eqref{eq:angle}, so there is
  exactly one solution in $2k \pi/\ln(b) < \Im(z) < (2\pi(k+1)
  -\pi)/\ln(b)$ for $k\neq 0,-1$.

  Let us now consider the case $k=0$. Here $f((0,\pi))=(-1,\infty)$,
  so if $c>1/e$ then $\ln(c)>-1$ and there is exactly one solution
  $f(x)=\ln(c)$ for $0<x<\pi$. If $\ln(c)\le -1$ then $f(x)=\ln(c)$ has
  no solution in $0<x<\pi$.

  We now consider the case $\alpha=2k\pi$, $k\ge 0$. If $k>0$ then
  \eqref{eq:angle} is invalid. So we consider $\alpha=0$ for which
  equation \eqref{eq:angle} is always valid. In this case only 
  $r=e^{cr}$ has to be satisfied. Now we look for zeros on $0\le
  x<\infty$ of the corresponding $g(x)=e^{cx}-x$. We can determine the
  global minimum $\mu$ at $x$ of this function by 
  \begin{align*}
    0&=g'(x)=ce^{cx}-1\\
    x&=\frac{1}{c}\ln\frac{1}{c}=-\frac{\ln(c)}{c}\\
    g''(x)&=c^2e^{cx}=c^2\frac{1}{c}=c>0\\
    \mu:=g(x)&=\frac{1}{c}(1+\ln(c))
  \end{align*}
  Clearly the minimum $\mu$ is smaller than 0 for $\ln(c)<-1$ and
  equal to $0$ for $\ln(c)=-1$ which corresponds to the last both
  cases $c=\frac{1}{e}$ having one solution and $0<c<\frac{1}{e}$
  having two solutions.
\end{proof}
\begin{proposition}[Repelling and attracting fixed points of $b^z$]
  Let $b>1$, then $\abs{{\exp_b}'(p)}>1$ for all non-real fixed
  points $p$. For the real fixed points in the
  case $1<b<e^{1/e}$ we have 
  \begin{align}
    {\exp_b}'(b^-) &< 1 & {\exp_b}'(b^+)&>1.
  \end{align}
  and ${\exp_b}'(e)=1$ for $b=e^{1/e}$.
\end{proposition}

\begin{proposition}
  Let $b>e^{1/e}$ and let
  \begin{align*}
    \log_{b,k}(z) &:= \frac{\log (z) + 2\pi i k}{\ln(b)}
  \end{align*}
  then for any $k>0$ and $z_0$ in the upper halfplane:
  \begin{align*}
    b[k]&=\lim_{n\to\infty} {\log_{b,k-1}}^{[n]}(z_0)\\
    {\exp_b}'(b[k]) &= \log_{b,k-1}(b[k]).
  \end{align*}
\end{proposition}
The fixed points can in the same way obtained as in knowledge
\ref{wk:fixedpoint:formula} via the branches of $\sfrt$, $\sfpw$ and
Lambert $W$, where the last is most important practically, as Lambert
$W$ is implemented (with its branches) in most computer algebra
systems
\begin{center}
\begin{tabular}{l|c|c|c}
  &Maple{\texttrademark} &Mathematica{\texttrademark} &Sage{\texttrademark}  \\\hline
       $W_k(z)=$ &LambertW(k,z) & ProductLog[k,z]& mpmath.lambertw(z,k)
\end{tabular}
\end{center}
With this branching we obtain for $k\ge 1$
\begin{align}
  &&b[k] &= \frac{W_{-k}(-\ln(b))}{-\ln(b)}&1&<b\\
  b^- &= \frac{W_{0}(-\ln(b))}{-\ln(b)} &
  b^+ &= \frac{W_{-1}(-\ln(b))}{-\ln(b)} & 
  1&<b\le\eta.
\end{align}

% Local Variables:
% TeX-master: "main"
% End:
